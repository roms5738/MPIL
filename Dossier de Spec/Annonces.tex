\section*{Annonces}
\addcontentsline{toc}{chapter}{Annonces}
La section “Annonces” permet à un utilisateur connecté de créer une annonce pour demander ou proposer un service, et de consulter les annonces pour les ajouter à ses favoris, ou contacter la personne ayant proposé l’annonce.
\subsubsection*{Accéder aux annonces}
\addcontentsline{toc}{section}{Accéder aux annonces}
Marc DUPOND, utilisateur connecté à l’application, veut accéder à la page “Annonces”. Il clique sur le bouton “Annonces” dans le menu à gauche accessible depuis toutes les pages. Il se trouve alors face à une fenêtre contenant la liste de toutes les annonces. 
\begin{center}
\includegraphics[width=14cm,height=8cm]{/Users/romain_baptiste/Desktop/Cours/FAC/M2/PIL/Maquettes/Annonce.png}
\emph{Liste des annonces}
\end{center}
\subsubsection*{Accéder à une annonce}
\addcontentsline{toc}{section}{Accéder à une annonce}
Marc DUPOND, utilisateur connecté à l’application, veut afficher une annonce. Il clique sur le bouton “Annonces” dans le menu à gauche accessible depuis toutes les pages, puis en sélectionne une dans la liste pour afficher le contenu avec les coordonnées de celui qui l’a posté.
\begin{center}
\includegraphics[width=14cm,height=8cm]{/Users/romain_baptiste/Desktop/Cours/FAC/M2/PIL/Maquettes/ExempleAnnonce.png}
\emph{Exemple d'une annonce}
\end{center}
\subsubsection*{Créer une annonce}
\addcontentsline{toc}{section}{Créer une annonce}
Marc DUPOND, utlisateur connecté à l’application, veut créer une annonce pour proposer un service. Il accède à la page “Annonces” et clique sur le bouton “proposer une annonce” accessible en bas de la rubrique “Liste des annonces”. Il est redirigé vers un formulaire où il indique le titre de l’annonce et son contenu. Après validation, l’annonce sera enregistrée dans la base de donnée et ajoutée à la liste des annonces.
\subsubsection*{Modifier une annonce}
\addcontentsline{toc}{section}{Modifier une annonce}
Marc DUPOND, utlisateur connecté à l’application, veut modifier une annonce qu’il a déja créé. Il accède à la page de son annonce, et clique sur le bouton “modifier” accessible que pour le créateur de l’annonce. Il est redirigé vers le formulaire de création avec les informations déjà enregistrées pour pouvoir les modifier. Après validation, les informations concernant cette annonce sont modifiées dans la base de données et les personnes ayant ajouté cette annonce dans leurs favoris recevront une notification les alertant du changement.
\subsubsection*{Supprimer une annonce}
\addcontentsline{toc}{section}{Supprimer une annonce}
Marc DUPOND, utilisateur connecté à l’application, veut modifier une annonce qu’il a déja créé. Il accède à la page de son annonce, et clique sur le bouton “supprimer” accessible que pour le créateur de l’annonce. Un message de confirmation apparaît, il valide. L’annonce sera également supprimée de la base de données, et de la liste des favoris de ceux qui l’ont ajoutée et seront aussi alertés de la suppression. Elle ne sera plus visible dans la liste des annonces.
\subsubsection*{Répondre à une annonce}
\addcontentsline{toc}{section}{Répondre à une annonce}
Marc DUPOND, utilisateur connecté à l’application, veut répondre à une annonce. Il accède à la page de cette annonce, et récupère les coordonnées de celui qui l’a proposé, affichées après le contenu de l’annonce.
\subsubsection*{Mémoriser une annonce}
\addcontentsline{toc}{section}{Mémoriser une annonce}
Marc DUPOND, utilisateur connecté à l’application, veut ajouter une annonce à ses favoris. Il accède à la page de cette annonce, et clique sur le lien “ajouter à mes favoris”. L’annonce sera enregistrée dans la liste des favoris dans la base de données et s’affichera dans le fil d’actualités de Marc.
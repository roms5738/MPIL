\section*{Associations}
\addcontentsline{toc}{chapter}{Associations}
La section “ Associations” permet à un utilisateur de créer, rejoindre et suivre les actualités d’une association étudiante.
\subsubsection*{Accéder aux associations} 
\addcontentsline{toc}{section}{Accéder aux associations}
Marc DUPOND, étudiant connecté à l’application, peut accéder à la page “Associations” en cliquant sur le bouton “Associations”. Il se trouve alors face à une fenêtre séparée en trois rubriques : “Mes associations”, “rechercher une association” et “nouvelles associations”. Deux actions sont également accessibles depuis les boutons correspondants au bas de la page : “créer une association” et “voir toutes les associations”.
Les rubriques “mes associations” et “nouvelles associations” se présentent sous forme de listes d’associations. Dans la première on retrouvera les associations dont on est membre. Dans la deuxième, il y aura la liste des dernières associations créées.
\begin{center}
\includegraphics[width=14cm,height=8cm]{/Users/romain_baptiste/Desktop/Cours/FAC/M2/PIL/Maquettes/Associations.png}
\emph{Liste des associations}
\end{center}
\subsubsection*{Accéder à une association}
\addcontentsline{toc}{section}{Accéder à une association}
Marc DUPOND, étudiant connecté à l’application, peut accéder à la page “Associations” en cliquant sur le bouton “Associations”. Il recherche ensuite l’association qui l’intéresse puis accède à la page en cliquant dessus. Il se retrouve face à une fenêtre séparée en trois rubriques : Les informations de l’association (nom, adresse postale, adresse mail), Une liste des membres de l’association et un fil d’actualité où il pourra suivre et commenter l’actualité de l’association.
\begin{center}
\includegraphics[width=13cm,height=7cm]{/Users/romain_baptiste/Desktop/Cours/FAC/M2/PIL/Maquettes/ExempleAssoc.png}
\\
\emph{Exemple d'une association}
\end{center}
\subsubsection*{Créer une association}
\addcontentsline{toc}{section}{Créer une association}
Marc DUPOND, étudiant connecté à l’application, veut créer une association afin de poster les actualités de son association étudiante et de faciliter la communication entre les différents membres. Il accède à la page des associations en cliquant sur le bouton “Associations” situé à gauche. Il est ensuite redirigé vers la page “associations”. Il clique ensuite sur “Créer une association” afin d’être redirigé vers un formulaire. Il indique le nom de l’association, une description, l’adresse mail, l’adresse postale puis valide. Après avoir validé l’association sera enregistrée dans la base de donnée, le créateur en sera administrateur et des personnes pourront s’y inscrire.
\subsubsection*{Rejoindre une association}
\addcontentsline{toc}{section}{Rejoindre une association}
Marc DUPOND, étudiant connecté à l’application, veut rejoindre une association afin d’être en contact avec ses membres et de suivre ses actualités. Il recherche l’association souhaitée via l’option de recherche (cf Recherche) puis accède à la page de l’association en cliquant sur dessus. Il clique ensuite sur le bouton “Rejoindre”. Il est maintenant membre de l’association et de son cercle de membres correspondant.
\subsubsection*{Supprimer une association}
\addcontentsline{toc}{section}{Supprimer une association}
Marc DUPOND, étudiant connecté à l’application et administrateur d’une association veut supprimer cette même association. Il accède à la page de l’association puis clique sur “Supprimer”. Un message de confirmation apparaît, il valide. L’association ne sera pas supprimée de la base de données mais la page ne sera plus accessible et le cercle des membres n’existera plus.
\subsubsection*{Quitter une association}
\addcontentsline{toc}{section}{Quitter une association}
Marc DUPOND, étudiant connecté à l’application et membre d’une association veut quitter cette association. Il accède à la page de l’association puis clique sur quitter. Un message de confirmation apparaît, il valide. L'utilisateur n’est désormais plus membre de l’association : il ne pourra plus suivre ses actualités et ne fait plus partie du cercle des membres. Cependant ses publications postées sur la page de l’association resteront présentes.
\subsubsection*{Poster une publication sur le fil d’actualités d’une association}
\addcontentsline{toc}{section}{Poster une publication sur le fil d’actualités d’une association}
Marc DUPOND, étudiant connecté à l’application et membre d’une association veut poster sur le fil d’actualités de cette même association. Il accède à la page de l’association, écrit un message dans le champ de texte correspondant puis valide. Son message sera désormais visible par tous les membres de l’association et ceux-ci pourront réagir.
\subsubsection*{Commenter une publication sur le fil d’actualités d’une association}
\addcontentsline{toc}{section}{Commenter une publication sur le fil d’actualités d’une association}
Marc DUPOND, étudiant connecté à l’application et membre d’une association veut commenter un message sur le fil d’actualités de cette même association. Il accède à la page de l’association, écrit un message dans le champ de texte correspondant(situé en dessous de la publication) puis valide. Son message sera désormais visible par tous les membres de l’association et ceux-ci pourront réagir.
\section*{Points d'intérêts}
\addcontentsline{toc}{chapter}{Points d'intérêts}
La section “ Points d’intérêts” permet à un utilisateur de consulter la liste des points d’intérêts de sa ville. L’utilisateur peut faire une demande à l’administrateur pour ajouter un point d’intérêts.
\subsubsection*{Accéder à la liste des points d’intérêts}
\addcontentsline{toc}{section}{SAccéder à la liste des points d’intérêts}
Marc DUPOND, utilisateur connecté, veut accéder à la liste des points d’intérêts. Il clique sur le bouton “Points d’intérêts” dans le menu à gauche accessible depuis toutes les pages.  Il se retrouve sur une page avec deux rubriques, l’une contenant ses points d’intérêts favoris, et une autre contenant la liste des points d’intérêts qu’il peut consulter.
\begin{center}
\includegraphics[width=13cm,height=7cm]{/Users/romain_baptiste/Desktop/Cours/FAC/M2/PIL/Maquettes/PI.png}
\\
\emph{Liste des points d'intérêts}
\end{center}
\subsubsection*{Accéder à un point d’intérêts}
\addcontentsline{toc}{section}{Accéder à un point d’intérêts}
Marc DUPOND, utilisateur connecté, veut consulter un point d’intérêts. Il clique sur le bouton “Points d’intérêts” dans le menu à gauche accessible depuis toutes les pages.  Il peut choisir le point qui l’intéresse dans la liste des ses favoris, ou celle de tous les points d’intérêts. Il clique sur le point d’intérêts choisi. Il est redirigé sur une page où s’affichera les informations de ce point d’intérêts contenant une description du lieu et un mur dans lequel il peut poster une publication ou en commenter.
\begin{center}
\includegraphics[width=13cm,height=7cm]{/Users/romain_baptiste/Desktop/Cours/FAC/M2/PIL/Maquettes/ExemplePI.png}
\\
\emph{Exemple d'un point d'intérêt}
\end{center}
\subsubsection*{Noter un point d’intérêt}
\addcontentsline{toc}{section}{Noter un point d’intérêt}
Marc DUPOND, étudiant à Nancy, vient de se rendre au “Magic Bowling”, il a beaucoup apprécié cette sortie. Il se connecte donc à l'application, et se rend sur la page “Magic Bowling” depuis la section de ses “Points d’intérêt favoris” de son fil d’actualité.
Marc appuie sur “Noter ce point d’intérêt”, et donne la note de 5 sur 5  et confirme. Cela va enclencher  le calcul de la note moyenne de ce point d'intérêt en prenant en compte la nouvelle note. Ce point d’intérêt se retrouve après cette modification dans le Top 5 des Point d’intérêts les mieux notés de Nancy, les étudiants l’ayant comme favoris seront alors notifiés de ce changement.


\subsubsection*{Mémoriser un point d’intérêt}
\addcontentsline{toc}{section}{Mémoriser un point d’intérêt}
Marc DUPOND, étudiant connecté à l'application, étant fan du Musée des Beaux Arts de Nancy, souhaite être informé lorsque que quelqu'un effectue une publication sur la page de ce point d’intérêt. Marc se rend donc sur la page du Musée des Beaux Arts de Nancy et appuie sur “Ajouter à mes favoris”, cela a pour effet d’ajouter dans la base de donnée et sur l'application le Musée des Beaux Arts de Nancy aux points d’intérêt favoris de Marc.
De plus cela va incrémenter le nombre de fois où ce point d’intérêt à été ajouté en tant que favori. Le Musée des Beaux Arts se retrouve maintenant parmi le Top 5 des points d’intérêt les plus appréciés,les étudiants l’ayant comme favoris seront alors notifié de ce changement.

\subsubsection*{Retirer un point d’intérêt de ses favoris}
\addcontentsline{toc}{section}{Retirer un point d’intérêt de ses favoris}
Marc DUPOND, étudiant connecté à l'application, à été déçu par la dernière exposition spéciale du Musée des Beaux Arts de Nancy, il ne veut plus en entendre parler. Marc se rend donc sur son fil d’actualité et dans la partie “Points d’intérêt favori”, il sélectionne le Muse des Beaux Arts de Nancy et choisit l’action “Supprimer de la liste”. Cela a pour effet d’enlever de la base de donnée et sur l'application le Musée des Beaux Arts de Nancy ds points d’intérêt favoris de Marc.
De plus cela va décrémenter le nombre de fois où ce point d’intérêt à été ajouté en tant que favori. Le Musée des Beaux Arts ne se retrouve maintenant plus parmi le Top 5 des points d’intérêt les plus appréciés,les étudiants l’ayant comme favoris seront alors notifiés de ce changement.


\subsubsection*{Poster un message sur mur}
\addcontentsline{toc}{section}{Poster un message sur mur}
Marc DUPOND, étudiant connecté à l'application, après s’être rendu au bar Mac Carty de Nancy souhaite informé les autres étudiants de l'application de l’arrivée d’une nouvelle bière à la carte du bar.
Il se rend sur la page du bar est sur le “mur” choisis l’action “Poster une publication”. Marc écrit alors la nouvelle et confirme. La base de donnée correspondante au mur est alors mise à jour et tous les étudiants de l'application ayant  comme favoris le MacCarty seront alors notifié de ce de cette publication.
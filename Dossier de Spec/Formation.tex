\section*{Formation}
\addcontentsline{toc}{chapter}{Formation}
La section “Formation" permet à un utilisateur de discuter, via des publications sur le mur de la formation, avec tous le étudiants inscrit dans sa formation.
\subsubsection*{Accéder à sa formation}
\addcontentsline{toc}{section}{Accéder à sa formation}
Marc DUPOND, étudiant connecté à l’application, peut accéder à la page “Formation” en cliquant sur le bouton "Formation". Il se trouve alors face à une fenêtre consacrée à sa formation où se trouve un mur où tous les étudiants inscrits a cette formation peuvent dialoguer via des publications et des commentaires. 
\begin{center}
\includegraphics[width=14cm,height=8cm]{/Users/romain_baptiste/Desktop/Cours/FAC/M2/PIL/Maquettes/Formation.png}
\emph{Mur d'une formation}
\end{center}
\subsubsection*{Poster une publication sur le mur d'une formation}
\addcontentsline{toc}{section}{Poster une publication sur le mur d'une formation}

Marc DUPOND, étudiant connecté à l’application veut poster un message sur le mur de sa formation. Il accède à la page de sa formation, écrit un message dans le champ de texte correspondant puis valide. Son message sera désormais visible par tous les utilisateurs de l’application inscrit à la même formation que lui.

\subsubsection*{Commenter une publication sur le mur d'une formation}
\addcontentsline{toc}{section}{Commenter une publication sur le mur d'une formation}

Marc DUPOND, étudiant connecté à l’application veut commenter un message sur le mur de sa formation. Il accède à la page de sa formtion, écrit un message dans le champ de texte correspondant(situé en dessous de la publication) puis valide. Son message sera désormais visible par tous les utilisateurs de l’application inscrits à sa formation.

\section*{Fil d'actualités}
\addcontentsline{toc}{chapter}{Fil d'actualités}
Cette partie est en rapport avec les informations de l’utilisateur, du fil d’actualité et de l’ajout des amis. Le profil permet de vérifier et de corriger ses informations personnelles. Le fil d’actualité permet d’être informé des dernières actualités des contacts de l’utilisateur.

\subsubsection*{Accéder au fil d’actualités}
\addcontentsline{toc}{section}{Accéder au fil d’actualités}
Marc DUPOND, après s’être connecté, accède directement à son fil d’actualités. Il peut aussi y accéder via le bouton “Fil d’actualités” accessible sur n’importe quelle page de l’application.
\\ 
\begin{center}
\includegraphics[width=14cm,height=8cm]{/Users/romain_baptiste/Desktop/Cours/FAC/M2/PIL/Maquettes/FilDactu.png}
\\
\emph{Fil d'actualités de Marc DUPOND}
\end{center}
\subsubsection*{Consulter Profil}
\addcontentsline{toc}{section}{Consulter Profil}
Marc DUPOND, étudiant connecté à l'application, veut consulter son profil afin de vérifier ses informations personnelles. Pour cela, il clique sur son nom qui est visible depuis n’importe quelle page. Ses informations personnelles apparaissent (nom, prénom, adresse mail, photo de profil, sexe, ses formations, ses associations, ses annonces).
\subsubsection*{Modifier Profil}
\addcontentsline{toc}{section}{Modifier Profil}
Marc DUPOND, étudiant connecté à l'application, veut modifier ses informations personnelles. Pour cela, il se rend sur son profil puis clique sur l’option de modification. Après sauvegarde par Marc, les informations seront effectives immédiatement.
\subsubsection*{Accéder fil d’actualités}
\addcontentsline{toc}{section}{Accéder fil d’actualités}
Marc DUPOND, étudiant connecté à l'application, veut accéder à son fil d’actualités, pour être au courant des dernières nouvelles de l'application. Pour cela, il clique sur le bouton “Fil d’actualités” présent sur le menu. Le fil d’actualités est composé de :
\begin{itemize}
\item ses notifications
\item ses suggestions
\item ses cours
\item ses évènements
\end{itemize}
Certaines notifications peuvent être interactive et proposer des options. Par exemple une demande d’ajout d’amis avec les options confirmer et refuser.
\subsubsection*{Modifier son fil d’actualités}
\addcontentsline{toc}{section}{Modifier son fil d’actualités}
Marc DUPOND, étudiant connecté à l'application, veut modifier les préférence de son fil d’actualités dans le but d’afficher uniquement ce dont il a envie. Pour cela il clique sur l’option « modifier le fil d’actualités ». Une sélection de type de suggestion et de notification est proposé. Ainsi que la possibilité de masquer les évènements et les cours.
\subsubsection*{Ajouter un amis dans le cercle d’amis}
\addcontentsline{toc}{section}{Ajouter un amis dans le cercle d’amis}
Marc DUPOND, étudiant connecté à l'application, veut ajouter un amis, pour avoir ses activités ajoutées à son fil d’actualités. Pour cela, il recherche la personne concernée avec l’outil de recherche, puis, une fois sur son profil utilise le bouton “ajouter comme amis”. Il devra attendre la confirmation de cet amis pour qu’il soit effectivement ajouté.
\subsubsection*{Se désinscrire}
\addcontentsline{toc}{section}{Se désinscrire}
Jean DUPOND, étudiant connecté à l'application, veut se désinscrire. Il clique sur le bouton "réglages", puis sur "se désinscrire". Un message de confirmation s'affiche, il accepte. Son profil ne sera désormais plus consultable, il disparaîtra des listes d'amis, des cercles, des évènements et des associations. Par contre, ses cours, publications et commentaires seront toujours visibles.
\subsubsection*{Faire une recherche}
\addcontentsline{toc}{section}{Faire une recherche}
Jean DUPOND, étudiant connecté à l'application, veut effectuer une recherche. Il rentre ses mots clefs sur la barre de recherche accessible depuis toutes les pages. Une recherche effectuée depuis la barre de recherche est réalisée parmi toutes les rubriques de l'application. Une fois la recherche effectuée, Jean pourra appliquer plusieurs filtres sur les résultats. Comme par exemple, afficher seulement les événements ou afficher les résultats dans la ville de Metz.
\section*{Cours}
\addcontentsline{toc}{chapter}{Cours}
La section “Cours” de l’application regroupe tout ce qui touche à l’enseignement. Il comprends les fonctionnalités concernant la mise en ligne et le partage de cours, de même que les différentes interactions que les utilisateurs peuvent avoir avec cette partie (rechercher, consulter, commenter, noter un cours).
\subsubsection*{Accéder à l’interface}
\addcontentsline{toc}{section}{Accéder à l’interface}
Marc DUPOND, étudiant connecté, souhaite accéder à la partie Cours de l’application. Pour cela, il clique sur le bouton “Cours” situé sur la gauche qui est accessible depuis n’importe quelle interface. Il se trouve alors face à une fenêtre séparer en quatre rubriques : “cours favoris”, “rechercher un cours”, “nouveaux cours” et “cours pertinents”. Deux actions sont également accessibles depuis les boutons correspondants au bas de la page : “créer un cours” et “demander un cours”.
	Les rubriques “cours favoris”, “cours pertinents” et “derniers cours” se présentent sous forme de listes de cours. Dans la première on retrouvera les cours que l’on aura marqué comme favoris (chaque inscrit possède sa liste de favoris). Dans la deuxième, il y aura la liste des derniers cours créés en adéquation avec ses intérêts (définis par la formation dans lequel il est inscrit, ses associations, et les cours qu’il a ajouté dans ses favoris ou bien notés). Et enfin, dans la troisième rubrique, la liste de cours corresponds aux cours qui auront été les mieux notés par les utilisateurs.
\begin{center}
\includegraphics[width=14cm,height=8cm]{/Users/romain_baptiste/Desktop/Cours/FAC/M2/PIL/Maquettes/Cours.png}
\emph{Liste des cours}
\end{center}
\begin{center}
\includegraphics[width=14cm,height=8cm]{/Users/romain_baptiste/Desktop/Cours/FAC/M2/PIL/Maquettes/ExempleCours.png}
\emph{Exemple d'un cours}
\end{center}
\subsubsection*{Créer un cours}
\addcontentsline{toc}{section}{Créer un cours}
Marc DUPOND, étudiant connecté, souhaite mettre en ligne un cours. Il accède à l’interface “cours” et clique sur le bouton “créer un cours”. Il est dirigé vers une autre interface où il devra remplir le formulaire affiché. Il renseigne le titre du cours, les divers mot-clés qui lui sont associés, il indique la rubrique dans laquelle ira s’inscrire le cours et enfin rédige le cours ou inscrit un lien vers celui-ci.
\subsubsection*{Consulter un cours}
\addcontentsline{toc}{section}{Consulter un cours}
Marc DUPOND, étudiant connecté, souhaite accéder à un cours. Pour cela, il doit cliquer sur un titre de cours, que ce dernier soit dans une sous-rubrique (“cours favoris”, “cours pertinents” ou “derniers cours”) ou qu’il y ait eu accès via la recherche de cours. Il est alors redirigé vers le mur correspondant au cours. Il y figure le cours demandé (ou le lien y correspondant) suivis des divers commentaires postés par les utilisateurs en réaction au cours. A partir de cette interface, il est également possible de noter, commenter ce cours ou de l’ajouter à ses favoris. 
\subsubsection*{Rechercher un cours}
\addcontentsline{toc}{section}{Rechercher un cours}
Marc DUPOND, étudiant connecté, souhaite rechercher un cours. Pour cela, il a deux solutions. Premièrement il peux effectuer sa recherche grâce au champ de recherche général, en y inscrivant les mots clés correspondants, il aura ainsi, en plus d’une liste de cours en rapport avec sa recherche, une liste de formations, d’associations et d'évènements. Deuxièmement, dans sa page “Cours” il y a un champ de recherche de cours ainsi qu’un bouton permettant d’accéder à la liste complète des cours.
\subsubsection*{Faire une publication sur un cours}
\addcontentsline{toc}{section}{Faire une publication sur un cours}
Marc DUPOND, étudiant connecté, est sur le mur d’un cours et souhaite, par exemple, laisser un mot pour demander de l’aide ou plus de détails sur un point du cours. Il lui suffit de créer une publication grâce au bouton “publier” qui lui affiche un champ texte à remplir. Sa publication apparaîtra alors sur la page et d’autres utilisateurs (pas forcément l’auteur) pourront soit commenter sa publication (répondre à sa question) soit ajouter simplement d’autres commentaires pour le cours. Les utilisateurs ayant commenté ce cours ou l’ayant pour favori recevront une notification qui les informera de la nouvelle activité sur le cours.
\subsubsection*{Commenter une publication sur un cours}
\addcontentsline{toc}{section}{Commenter une publication sur un cours}
Marc DUPOND, étudiant connecté, est sur le mur d’un cours et un autre étudiant a publié sur ce cours une question et Marc voudrait réagir. Il lui suffit de cliquer sur “commenter cette publication” et il pourra laisser un commentaire ( de préférence utile) sur cette publication en remplisssant le champ texte à disposition. Les utilisateurs ayant un rapport avec ce cours (auteur, publication, commentaire de plublication ou favori) recevront une notification qui les informera de la nouvelle activité sur le cours.
\subsubsection*{Noter un cours}
\addcontentsline{toc}{section}{Noter un cours}
Marc DUPOND, étudiant connecté, est sur le mur d’un cours et souhaite lui donner une note. Il clique donc sur le bouton “noter ce cours”, il entre ensuite la note (entre 0 et 10) dans le champs texte qui est apparue et valide sa note.
\subsubsection*{Demander un cours}
\addcontentsline{toc}{section}{Demander un cours}
Marc DUPOND, étudiant connecté, ne trouve pas le cours qu’il souhaite sur l’application. Il peut alors effectuer une demande à l’administrateur en précisant le sujet du cours et son domaine via le bouton qui sera présent sur l’interface générale du “Cours”. Un mail sera envoyé à l’administrateur qui choisira alors de donner suite ou non à la demande.
\subsubsection*{Mémoriser un cours}
\addcontentsline{toc}{section}{Mémoriser un cours}
Marc DUPOND, étudiant connecté, a lu un cours qui l’a intéressé et qu’il souhaite retrouver plus facilement par la suite. Il se rend sur le mur du cours en question (via l’outil “recherche” ou les listes de cours) puis il clique sur le bouton “ajouter ce cours à mes favoris”. Ceci fait, le cours sera accessible depuis l’interface principale de la catégorie “Cours” dans la sous-rubrique “Mes cours favoris”.
\subsubsection*{Retirer un cours de ses favoris}
\addcontentsline{toc}{section}{Retirer un cours de ses favoris}
Marc DUPOND, étudiant connecté, avait suivi un cours de “Prolog” qui ne lui servira plus à rien, il souhaite retirer le cours qui est dans ses favoris. Il accède à la rubrique des cours, sélectionne le cours en question et choisit l’action “Supprimer de mes favoris”. Il n’apparaîtra plus dans la liste. De plus cela va décrémenter le nombre de fois où ce cours à été ajouté en tant que favori. Si le cours “Prolog” ne se trouve alors plus dans le top 5 (on peut rêver, il y sera peut-être un jour), les utilisateurs l’ayant comme favori recevront une notification pour les informer du changement.
\section*{Vocabulaires}
\addcontentsline{toc}{chapter}{Vocabulaires}
\subsubsection*{Mur}
\addcontentsline{toc}{section}{Mur}
Un mur offre la possibilité aux étudiants de réagir par le biais de commentaires. Il y a un mur pour chaque formation, chaque évènement, chaque association, chaque cours, et chaque point d’intérêt. 
\subsubsection*{Fil d’actualités}
\addcontentsline{toc}{section}{Fil d’actualités}
Le fil d’actualité est propre à chaque étudiant inscrit sur l'application. Il regroupe toutes les informations que l’étudiant souhaite afficher. Le fil d’actualité peut contenir les cours favoris de l'étudiant, les prochains évènements auxquels il a prévu de participer, les association qu’il a rejoint et ses annonces favorites. De plus le fil d’actualités contient les notifications. Si un autre étudiant souhaite l’inviter dans sa liste d’amis ou alors qu’un évènement change de date, alors ces information seront afficher sur le fil d’actualité de l’étudiant. 
\subsubsection*{Cercles d’amis et liste d’amis}
\addcontentsline{toc}{section}{Cercles d’amis et liste d’amis}
Il y a deux types de cercle d’amis. Un cercle d’amis propre à chaque formation et un cercle propre à chaque ville. En effet à chaque fois qu’un étudiant s’inscrit sur l'application et qu’il renseigne sa formation alors il sera automatiquement ajouté au cercle d’amis de la formation et lorsqu’il renseigne sa ville il sera ajouter au cercle d’amis de la ville. Les cercles permettent de proposer aux étudiants des évènements, des cours, des associations ou des points d’intérêts. Par exemple si plusieurs étudiants de la formation Master 2 informatique à Nancy participent à une conférence sur JAVA alors la conférence sera proposée au reste du cercle d’amis de la formation Master 2 informatique. 
Chaque étudiant aura également une liste d’amis, où il pourra ajouter n’importe quel étudiant,toutes villes confondues,  inscrit sur l'application.
\subsubsection*{Profils}
\addcontentsline{toc}{section}{Profils}
Le profil d’un étudiant inscrit contient toutes ses informations personnelles. Il est consultable par n’importe quel étudiant inscrit sur l'application. 
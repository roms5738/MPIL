\section*{Évènements}
\addcontentsline{toc}{chapter}{Évènements}
La section “Évènements” permet à un utilisateur de créer, rejoindre et suivre les actualités d’un évènement. Il peut s’inscrire à l’évènement pour être au courant des actualités liées à cet évènement. Il peut également se désinscrire. 
\subsubsection*{Accéder aux évènements}
\addcontentsline{toc}{section}{Accéder aux évènements}
Marc DUPOND, étudiant connecté à l’application, peut accéder à la page “Évènements” en cliquant sur le bouton “Évènements”. Il se trouve alors face à une fenêtre séparée en trois rubriques : “Mes évènements”, “rechercher un évènement” et “nouveaux évènements”. Deux actions sont également accessibles depuis les boutons correspondants au bas de la page : “créer un évènement” et “voir toutes les évènements”.
	Les rubriques “mes évènements” et “nouveaux évènements” se présentent sous forme de listes d’évènements. Dans la première on retrouvera les évènements aux quels on participe. Dans la deuxième, il y aura la liste des derniers évènements créés.
\begin{center}
\includegraphics[width=14cm,height=8cm]{/Users/romain_baptiste/Desktop/Cours/FAC/M2/PIL/Maquettes/Evenements.png}
\emph{Liste des évènements}
\end{center}
\subsubsection*{Accéder à un évènement}
\addcontentsline{toc}{section}{Accéder à un évènement}
Marc DUPOND, étudiant connecté à l’application, peut accéder à la page “Évènements” en cliquant sur le bouton “Évènements”. Il recherche ensuite l’évènement qui l’intéresse puis accède à la page en cliquant dessus. Il se retrouve face à une fenêtre séparée en trois rubriques : Les informations de l’évènement (nom, adresse,description, date), Une liste des membres participant à l’évènement et un fil d’actualité où il pourra suivre et commenter l’actualité de l’évènement.
\begin{center}
\includegraphics[width=13cm,height=7cm]{/Users/romain_baptiste/Desktop/Cours/FAC/M2/PIL/Maquettes/ExempleEvenement.png}
\\
\emph{Exemple d'un évènement}
\end{center}
\subsubsection*{Créer un évènement}
\addcontentsline{toc}{section}{Créer un évènement}
Marc DUPOND, étudiant connecté à l’application, veut créer un évènement afin de faciliter la communication entre les différents participants et leur permettre de voir facilement les informations (lieu, date etc…). Il accède à la page des évènements en cliquant sur le bouton “Évènements” situé à gauche. Il est ensuite redirigé vers la page “évènements”. Il clique ensuite sur “Créer un évènement” afin d’être redirigé vers un formulaire. Il indique le nom de l'évènement, une description, l’adresse postale où aura lieu l’évènement puis valide. Après avoir validé l’évènement sera enregistré dans la base de donnée,le créateur en sera administrateur et des personnes pourront y participer.
\subsubsection*{Participer à  un évènement}
\addcontentsline{toc}{section}{Participer à  un évènement}
Marc DUPOND, étudiant connecté à l’application, veut participer à  un évènement afin de pouvoir suivre l’actualité et les éventuels modifications apportées à l'évènement. Il recherche l'évènement souhaité via l’option de recherche (cf Recherche) puis accède à la page de l'évènement en cliquant dessus. Il clique en suite sur le bouton “Participer”. Il est maintenant enregistré en tant que participant à l'évènement et pourra suivre les actualités de celui-ci (changement de date, annulation etc…)
\subsubsection*{Supprimer un évènement}
\addcontentsline{toc}{section}{Supprimer un évènement}
Marc DUPOND, étudiant connecté à l’application et modérateur d’un évènement veut supprimer ce même évènement. Il accède à la page de l’évènement puis clique sur “Supprimer”. Un message de confirmation apparaît, il valide. L’évènement ne sera pas supprimé de la base de données mais la page ne sera plus accessible.
\subsubsection*{Commenter une publication}
\addcontentsline{toc}{section}{Commenter une publication}
Marc DUPOND, étudiant connecté à l’application, veut commenter une publication postée sur la page d’une association ou d’un évènement. Il accède à la page de l’association ou de l’évènement afin de pouvoir voir la publication. Il tape son commentaire dans la zone de texte située en dessous de la publication puis valide. Le commentaire est enregistré dans la base de données et sera visible par tous les membres de l’association ou de l’évènement.
\subsubsection*{Quitter un évènement}
\addcontentsline{toc}{section}{Quitter un évènement}
Marc DUPOND, étudiant connecté à l’application et membre d’un évènement veut quitter cet évènement. Il accède à la page de l’évènement puis clique sur quitter. Un message de confirmation apparaît, il valide. L’utilisateur n’est désormais plus un participant de l’évènement : il ne pourra plus suivre ses actualités. Cependant ses publications postées sur la page de l’association resteront présentes.
\subsubsection*{Poster une publication sur le mur d’un évènement}
\addcontentsline{toc}{section}{Poster une publication sur le mur d’un évènement}
Marc DUPOND, étudiant connecté à l’application veut poster sur le mur d’un évènement. Il accède à la page de l’évènement, écrit un message dans le champ de texte correspondant puis valide. Son message sera désormais visible par tous les utilisateurs de l’application visitant la page de l’évènement correspondant et ceux-ci pourront réagir.
\subsubsection*{Commenter une publication sur le mur d’un évènement}
\addcontentsline{toc}{section}{Commenter une publication sur le mur d’un évènement}
Marc DUPOND, étudiant connecté à l’application veut commenter un message sur le mur d’un évènement. Il accède à la page de l’évènement, écrit un message dans le champ de texte correspondant(situé en dessous de la publication) puis valide. Son message sera désormais visible par tous les utilisateurs de l’application visitant la page de l’évènement correspondant et ceux-ci pourront réagir.